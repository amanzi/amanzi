%!TEX root = theory.tex
% =========================================================================
% -------------------------------------------------------------------------
% Single and Multiphase Flow:
% -------------------------------------------
%
%  This is a good place to outline key objectives of this section.
%
% -------------------------------------------------------------------------

\def\bnabla{{\boldsymbol{\nabla}}}
\def\bg{{\boldsymbol{g}}}
\def\bq{{\boldsymbol{q}}}
\def\K{{\mathbb K}}

% units
\def\ucdot{{\,\cdot\,}}
\def\ukg{{\rm kg}}
\def\um{{\rm m}}
\def\us{{\rm s}}

\section{Flow Processes}         
\label{sec:flow}

\subsection{Overview}

Subsurface flow simulations typically assume that Darcy's law is
valid. As this law gives a relationship between velocity and pressure,
it essentially replaces the momentum equation. There has been much
research to support the validity of Darcy's
Law~\citep{bear-1972}.
Most references give the applicability of
Darcy's Law to be for laminar flows with Reynolds numbers less that 10 using the pore throat diameter for a soil.
There has been some effort to include inertial as well as turbulence effects that can occur near the wells.

It is also assumed that thermodynamic equilibrium (mechanical and
thermal) exists for each grid block.  Sub-grid scale features will
often play a prominent role in multi-fluid simulations. Faults and
fractures will likely be fast paths for contaminant transport and can
effectively be treated with multiple porosity models. Similarly,
rate-limited diffusion from clay inclusions can also be modeled with a
multiple porosity material.


%-------------------------------------------------------------------------------------
% Single-Phase Flow 
%-------------------------------------------------------------------------------------

\subsection{Single-Phase Flow}
\label{sec:single-phase-flow}

\subsubsection{Overview}
\label{sec:single-phase-overview}

The most basic flow model is a single-phase flow in a porous medium.  
Notwithstanding its simplicity, it has a wide application to
describing subsurface processes.

\subsubsection{Assumptions and Applicability}

There are many assumptions required for the strict validity of Darcy's
Law, including incompressible and laminar flow.


\subsubsection{Process Model Equations}

The governing equations for the single phase flow in porous media 
under isothermal conditions are
\begin{equation}
  \phi (s_s + s_y) \frac{\partial p_l}{\partial t} 
  =
  \boldsymbol{\nabla} \cdot (\rho_l \boldsymbol{q}_l) + Q,
  \qquad
  \bq_l = -\frac{\K}{\mu} 
  (\bnabla p_l - \rho_l \bg),
\end{equation}
where $\phi$ is porosity [-],
$s_s$ and $s_y$ are specific storage and specific yield, respectively,
$\rho_l$ is fluid density [$\ukg \ucdot \um^3$],
$Q$ is source or sink term [$\ukg \ucdot \um^3 \ucdot \us$],
$\bq_l$ is the Darcy velocity [$\um \ucdot \us$],
and $\bg$ is gravity vector [$\um \ucdot \us^2$].


%-------------------------------------------------------------------------------------
% Richards (variably saturated) Flow
%-------------------------------------------------------------------------------------

\subsection{Richards Equation}
\label{sec:richards-equation}

\subsubsection{Overview}
\label{sec:richards-overview}

\subsubsection{Process Model Equations} 
\label{sec:richards-model-equations}

\begin{equation}
  \frac{\partial \theta}{\partial t} 
  = \bnabla \cdot (\eta_l \bq_l) + Q,
  \qquad
  \bq_l = -\frac{\K k_r}{\mu} 
  (\bnabla p - \rho_l \bg)
\end{equation}

\subsubsection{Capillary Pressure -- Saturation Relations}  
\label{sec:pc_s_relations}

\subsubsection{Relative Permeability Relations}  
\label{sec:RelativePerm}

Bla-bla-bla

%-------------------------------------------------------------------------------------
% Thermal Richards (variably saturated) Flow with water vapor
%-------------------------------------------------------------------------------------

\subsection{Thermal Richards Equation}
\label{sec:thermal-richards-equation}

\subsubsection{Overview}
\label{sec:thermal-richards-overview}

\subsubsection{Process Model Equations} 
\label{sec:thermal-richards-model-equations}

\begin{equation}
  \frac{\partial \theta}{\partial t} 
  =
  \bnabla \cdot (\eta_l \bq_l)
  - \bnabla \cdot (\K_g \bnabla \big(\frac{p_v}{p_g}\big)) + Q,
  \qquad
  \bq_l = -\frac{\K k_r}{\mu} (\bnabla p - \rho_l \bg)
\end{equation}

%-------------------------------------------------------------------------------------
% Richards (variably saturated) Flow with dual porocity submodel
%-------------------------------------------------------------------------------------

\subsection{Richards Equation with Dual Porosity Submodel}
\label{sec:dual-porosity-richards-equation}

\subsubsection{Overview}
\label{sec:sual-porosity-richards-overview}

\subsubsection{Process Model Equations} 
\label{sec:sual-porosity-richards-model-equations}
