% =========================================================================
% -------------------------------------------------------------------------
% Introduction:
% -------------------------------------------
%
%  This is a good place to outline key objectives of this section.
%
% -------------------------------------------------------------------------

\section{Introduction to Process Models Requirements}

\subsection{ASCEM Overview}

The Advanced Simulation Capability for Environmental Management
(ASCEM) is intended to be a state-of-the-art scientific tool and
approach for understanding and predicting contaminant fate and
transport in natural and engineered systems.  
%This modular and open
%source high performance computing capability will facilitate
%integrated approaches to modeling and site characterization that
%enable robust and standardized assessments of performance and risk for
%EM cleanup and closure activities. 
The ASCEM program is aimed at addressing critical EM program needs to
better understand and quantify flow and contaminant transport behavior
in complex geological systems.  It will also address the long-term
performance of engineered components including cementitious materials
in nuclear waste disposal facilities, in order to reduce uncertainties
and risks associated with {DOE EM's} environmental cleanup and closure
activities.  Building upon national capabilities developed from decades
of Research and Development in subsurface geosciences, computational
and computer science, modeling and applied mathematics, and
environmental remediation, the ASCEM initiative will develop an
integrated, open-source, high-performance computer modeling system for
multiphase, multicomponent, multiscale subsurface flow and contaminant
transport.  This integrated modeling system will incorporate
capabilities for predicting releases from various waste forms,
identifying exposure pathways and performing dose calculations, and
conducting systematic uncertainty quantification. The ASCEM approach
will be demonstrated on selected sites, and then applied to support the
next generation of performance assessments of nuclear waste disposal
and facility decommissioning across the EM complex.
%\todo{It seems like we have should include something about the
%   time-scale here}

The Multi-Process High Performance Computing (HPC) Simulator is one of
three thrust areas in ASCEM.  The other two thrusts are the Platform
and Integrated Toolsets (dubbed the Platform) and Site Applications.
The primary objective of the HPC Simulator thrust is to provide a
flexible and extensible computational engine, named Amanzi, to
simulate the coupled processes and flow scenarios described by the
conceptual models developed with the Platform toolset, named Akuna.
The graded and iterative approach to assessments naturally generates a
suite of conceptual models that span a range of process complexity,
potentially coupling hydrological, biogeochemical, geomechanical, and
thermal processes.  Akuna will use ensembles of these simulations to
quantify the associated uncertainty, sensitivity, and risk.  The
Process Models task within the HPC Simulator thrust focuses on the
mathematical descriptions of the relevant physical processes.

\subsection{Purpose and Scope of this Document}

At the highest level of the HPC Simulator design is a set of process
models that mathematically represent the physical, chemical, and
biological phenomena controlling contaminant release into, and
transport in, the subsurface.  The objective of this requirements
document is to provide a catalogue of process models, along with their
detailed mathematical formulation, for potential implementation in the
HPC Simulator.  This concise mathematical description and accompanying
analysis, provides critical information for requirements and design of
both the HPC Core Framework (Task 1.1.2.2) and the HPC Toolsets (Task
1.1.2.3). This effort will also leverage ongoing efforts elsewhere in
the Department of Energy, including the Cementitious Barrier Project
(CBP) funded also by Environmental Management through EM-31.

It is important to note that with its focus on mathematical
descriptions for a catalogue of process models, this requirements
document is significantly different from a traditional Software
Requirements Specification (SRS) document.  Hence, this document does
not follow the IEEE Std 830-1998 template, and instead uses a process
category based layout that is summarized in Section~\ref{sec:layout}.
Moreover, this mathematical focus serves multiple audiences:
%
\begin{enumerate}
\item This document provides guidance to the developers engaged in
  designing and implementing the HPC Core Framework and HPC Toolsets.
  To meet their needs, sufficient detail for each process model is
  provided in the form of a background discussion, supporting equations, 
  and references to relevant literature.  The process models that are
  currently targeted for implementation in the Phase II demonstration 
  are listed in
  Table~\ref{tab:phase-ii-demo-models}.
\item The document is also intended for ``domain scientists'' whose
  primary interest is in the processes themselves.  The presentation
  is intended to justify the choice of process models and their
  mathematical detail.
\item Finally, the document is also intended for end users engaged in
  individual site applications.
\end{enumerate}
%
Over time this document will evolve into a comprehensive graded
presentation of models, from complex to simple, under a general
mathematical framework for each process category.  The prioritization
and selection of process models from this catalogue for implementation
is discussed in Section~\ref{sec:selecting-process-models}.
%
%~\todo{Not sure what categories you are referring to here. -Finsterle}
Evolution of the list of processes is inevitable, and the modular
design of the HPC Simulator will easily accommodate the addition of
new process implementations. 


\subsection{Links to Other Thrust Areas}
\label{sec:other-thrusts}

The set of requirements detailed in this document have a clear
connection to the other two thrust areas in ASCEM.  In particular,
many of the process model requirements follow from recommendations
made from end users as captured in the User Suggestions for ASCEM
Requirements Documents (Section 5, pp. 18--19)~\citep{ASCEM-SiteApps}.
%These requirements include the need for multiphase flow and transport,
%reactive transport, and both groundwater and surface water
%flow.
These requirements include the need for multiphase flow and transport,
reactive transport, and both groundwater flow.  Surface water flow, an important process at some contaminated sites, will be added as needed at a later date, but is currently beyond the scope of the present effort.
%
%~\todo{wasn't this excluded? -Finsterle} 
%
In addition, there is a need for modeling the degradation of
engineered barriers, including, for example, covers, liners,
cementitious materials and waste forms.  Other aspects that were
deemed desirable include the modeling of radionuclides, source
releases, and fractured media.  All of these needs have been
recognized and addressed in some form within this document.

Similarly, there are several areas where the process models will need
to interface with the Platform and Integrated Toolsets Thrust Area.
The Platform Thrust Area is tasked with developing the tools necessary
to set up the conceptual models.  These models represent the
subsurface flow and reactive transport processes described in this
document and will include 1) the geologic setting or framework, 2) the
physical, geochemical, and biological processes and their
interactions, and 3) the scenarios to which the models are applied.
An end user wishing to perform a simulation will have to specify
features such as the process models, the initial and boundary
conditions, and the material properties.  These features will have to
be consistent with and fully integrated with the existing capabilities
of the process models.  For example, for each process model that is
available, there will be an associated set of constitutive models from
which a user can select.  In turn, each process model will have a set
of specifications that will dictate the parameter requirements that
are specific to the particular model and associated constitutive
relations.  Because of these links, there needs to be close ties
between the development of the process models and the tools being
developed by the Platform Thrust area so that the final simulation
package can be reliable, robust, and easy to use.  
%
%~\todo{Need a diagram showing the overall ASCEM framework and a
%  diagram which depicts the interfaces of the thrust areas.}



\subsection{Organization and Layout of this Document}
\label{sec:layout}

The remainder of this document is organized based on individual process category.
These include presentation of
\emph{Isothermal Flow Processes} in Section~\ref{sec:flow-processes}, 
\emph{Thermal Processes} in Section~\ref{sec:thermal-processes}
\emph{Transport Processes} in Section~\ref{sec:transport-processes}, and 
\emph{Biogeochemical Reaction Processes} in Section~\ref{sec:biogeochemical}. 
%\emph{Colloid Transport Processes} in Section~\ref{sec:colloids}, 
%\emph{Thermal Processes} in Section~\ref{sec:thermal}, 
%\emph{Geomechanical Processes} in Section~\ref{sec:geomechanics}, and 
%\emph{Source Terms}, in Section~\ref{sec:source-terms}.  
%
Each of these process categories may present multiple process models.  
%For example, biogeochemistry includes individual process models for sorption, 
%mineral precipitation-dissolution, microbially-mediated reactions, colloid
%generation, etc.  
%In addition, models of differing fidelity can be accommodated by the HPC code so, 
%for example, several models for sorption 
%(e.g., classical Kd, multicomponent-multisite ion exchange,
%non-electrostatic surface complexation, electrostatic surface
%complexation) are included.


In Section~\ref{sec:flow-processes}, \emph{Isothermal Flow Processes}, 
we consider models of fluid flow at constant temperature.
We limit ourselves to the case of a single fluid phase, i.e. no water-oil flows.
Based on the saturation levels we use one of two models:
\emph{Darcy's Law} for fully saturated flows and
\emph{Richards Equation} for partially saturated flows.
Further, use of Richards Equation requires certain assumptions on the gas phase not moving.

In Section~\ref{sec:thermal-processes}, \emph{Thermal Processes},
we discuss extension of the models of Section~\ref{sec:flow-processes}
to the case of changing temperature by adding a concervation of energy equation.


In Section~\ref{sec:transport-processes}, \emph{Transport Processes},
we consider models for transport of solute species 
that can be part of either gas, fluid or solid phase.
In the solid phase specied do not move.
In the gas phase species can only diffuse (due to the assumptions of Richards equation),
In the fluid phase the transport of the species is affected by the 
fluid flow as well as dispersion and diffusion processes. 
In this section we also consider a dual porocity models,
designed to capture the difference in fluid motion in the cracks and pores.  
 

In Section~\ref{sec:biogeochemical}, \emph{Biogeochemical Reaction Processes},
we consider a variety of biochemical reaction processes,
which which from the point of view of model equations
convert one species into others and result in various heat sources.
  

The difficult question of prioritizing and ultimately selecting process
models for implementation in any given year is addressed in the
following subsection (Section~\ref{sec:selecting-process-models}).
Finally, the notational conventions and variables are summarized in
Section~\ref{sec:notations}.


\subsection{Prioritization and Selection of Process Models}
\label{sec:selecting-process-models}

Within each section that focuses on a particular process category
(e.g., flow, transport), sub-sections contain the details of specific
process models that may be implemented in the HPC Simulator.
However, the inclusion of a process model in this document does not
guarantee it will be implemented in the HPC Simulator in the near
term.  A combination of many factors will be considered in the
prioritization of process models and selection for implementation in
any given year.  For example, these factors include, the relative
importance of the process models to various EM sites and the
availability of required data for EM sites of interest.  These types
of information are gathered and organized by the Site Application
Thrust.  In the first years of the ASCEM project, prioritization must also
consider the overhead of initiating development of the supporting HPC
Core Framework (i.e., supporting infrastructure), as well as the design and 
initial development of the HPC Toolsets (i.e., the fundamental algorithmic
building blocks).

Given these constraints, discussions involving all three thrusts were
necessary to generate an initial list of process models that would be
supported by Amanzi for the Phase~I demonstration. This list is
updated each year through further discussion with all three thrusts to
include additional process models that are needed for the next ASCEM
demonstration, or at desired sites.  The current list of process
models supported by Amanzi, which will support the Phase~II
demonstration, is presented in Table~\ref{tab:phase-ii-demo-models}.

%
% \todo[color=cyan]{GEH: Single-phase $\rightarrow$ Single-phase
%   saturated, Richards $\rightarrow$ Single-phase
%   variably-saturated?}
%
\begin{table}[ht!]
\caption{The process models that will be supported
  by the HPC Simulator, Amanzi, for the Phase~II demonstration 
  are listed along with their corresponding subsection.
  \label{tab:phase-ii-demo-models}
}
\begin{center}
\begin{tabular}{lll}
   Process Category &   Process Model   &     Section       \\
  \hline
  \hline
  \hspace*{2in} & \hspace*{2in} & \\[-10pt]
  Flow                       &  Single-phase saturated
                             &  \ref{sec:flow-single-phase} 
  \\
  Flow                       &  Single-phase variably saturated (Richards)
                             &  \ref{sec:richards-equation}      
  \\
  \hline           
  & & \\[-10pt]
  Transport                  &  Advective
                             &        \ref{sec:transport-advection}
  \\
  Transport                  &  Dispersive
                             &        \ref{sec:transport-dispersion}
  \\
  Transport                  &  Diffusive
                             &        \ref{sec:transport-diffusion}
  \\
  Transport                  &  Dual porosity (cracks-pores)
                             &         \ref{sec:transport-single-phase-dual-porosity}
  \\
  \hline           
  & & \\[-10pt]
  Biogeochemical Reactions   &  Linear distribution coefficient ($K_d$)
                             &   \ref{sec:Sorption}      
  \\
  Biogeochemical Reactions   &  Electrostatic and non-electrostatic 
                                surface complexation           
                             &   \ref{sec:Sorption}      
  \\
  Biogeochemical Reactions   &  Multicomponent ion exchange           
                             &   \ref{sec:Sorption}      
  \\
  Biogeochemical Reactions   &  Precipitation/Dissolution  
                             &   \ref{sec:mineralPrecipDissolution}
  \\
  Biogeochemical Reactions   &  Radioactive decay + ingrowth
                             &   \ref{sec:AqueousComplexation}
  \\
  \hline
  \hline
\end{tabular}
\end{center}
\end{table}

Finally, it is important to note that the set of process models described
in this document is by no means exhaustive.  Process models will be
added based on the prioritization of specific EM needs during periodic reviews 
and updates of this document.



%--------------------------------------
\subsection{Variables and Notations}
\label{sec:notations}


% bold symbols
\def\bnabla{{\boldsymbol{\nabla}}}
\def\bg{{\boldsymbol{g}}}
\def\bq{{\boldsymbol{q}}}
\def\bx{{\boldsymbol{x}}}
\def\bJ{{\boldsymbol{J}}}
\def\K{{\mathbb K}}

% abbreviations
\def\Frac{\displaystyle \frac}

% units
\def\ucdot{{\,\cdot\,}}
\def\ukg{{\rm kg}}
\def\um{{\rm m}}
\def\us{{\rm s}}
\def\umol{{\rm mol}}
\def\upa{{\rm Pa}}


In this section we list all the variables used throughout the rest of the document.
In particular for multiple variables we use subscript 
$l$, $m$, $f$ or $r$ to indicate that the particular quantity 
is the property of the  
\emph{liquid}, \emph{matrix}, \emph{fracture} or \emph{rock}, respectively.
The difference between the matrix and the rock is in that matrix will typically refer 
to the voids that can be filled with liquid or gas while the rock is the solid.


\begin{center}
\begin{longtable}{cp{7cm}c}
\caption{List of global variables.} \label{table:flow-list-of-variables} \\

\multicolumn{1}{c}{Symbol} & \multicolumn{1}{c}{Meaning} & \multicolumn{1}{c}{Units} \\
\hline  \hline 
\endfirsthead

\multicolumn{3}{c}{{\tablename} \thetable{} -- Continued} \\
\multicolumn{1}{c}{Symbol} & \multicolumn{1}{c}{Meaning} & \multicolumn{1}{c}{Units} \\
\hline  \hline 
\endhead

\hline \multicolumn{3}{c}{{Continued on next page}} \\ 
\hline \hline 
\endfoot

\hline \hline
\endlastfoot

$C_i$      & concentration of $i$th species    &  $\umol\ucdot\um^{-3}$ \\
$h$        & hydrolic head        &  $\um$  \\
$h_l$      & enthalpy of the liquid        &  $\um$  \\
$\bg$      & gravity vector       &  $\um\ucdot\us^{-2}$  \\
$g$        & gravity magnitude    &  $\um\ucdot\us^{-2}$  \\
$\bJ$      & flux                 &  $\umol\ucdot\um^{-2}\ucdot\us^{-1}$  \\
$\K$       & absolute permeability tensor & $\um^2$ \\
$k_{rl}$   & relative permeability&  $-$ \\
$p_l$      & liquid pressure      &  $\upa$ \\
$\bq$      & Darcy velocity       &  $\um\ucdot\us^{-1}$  \\
$S_s$      & specific storage     &  $\um^{-1}$  \\
$S_y$      & specific yield       &  $-$  \\
$s_l$      & liquid saturation    &  $-$ \\
$u_l$      & internal energy of the liquid   &  $\ukg\cdot\um^2\cdot \us^{-2}$ \\
$u_r$      & internal energy of the rock     &  $\ukg\cdot\um^2\cdot \us^{-2}$  \\
\hline
$\mu_l$    & liquid viscosity     &  $\upa\ucdot\us$ \\
$\rho_l$   & liquid density       &  $\ukg\ucdot\um^{-3}$ \\
$\phi$     & porosity             &  $-$  \\
$\phi_f$   & porosity of fracture &  $-$  \\
$\phi_m$   & porosity of matrix   &  $-$  \\
$\eta_l$   & molar liquid density &  $\umol\ucdot\um^{-3}$ \\
$\theta$   & volumetric water content  &  $\umol\ucdot\um^{-3}$ \\
$\theta_f$ & volumetric water content of fracture &  $\umol\ucdot\um^{-3}$ \\
$\theta_m$ & volumetric water content of matrix   &  $\umol\ucdot\um^{-3}$ \\

\end{longtable}
\end{center}








