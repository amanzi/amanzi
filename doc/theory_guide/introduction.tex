% =========================================================================
% -------------------------------------------------------------------------
% Introduction:
% -------------------------------------------
%
%  This is a good place to outline key objectives of this section.
%
% -------------------------------------------------------------------------

\section{Introduction to Process Models Requirements}

\subsection{Overview}

NEEDS TO BE WRITTEN


\subsection{Purpose and Scope of this Document}

At the highest level of the HPC Simulator design is a set of process
models that mathematically represent the physical, chemical, and
biological phenomena controlling contaminant release into, and
transport in, the subsurface.  The objective of this requirements
document is to provide a catalogue of process models, along with their
detailed mathematical formulation, for potential implementation in the
HPC Simulator.  This concise mathematical description and accompanying
analysis, provides critical information for requirements and design of
both the HPC Core Framework (Task 1.1.2.2) and the HPC Toolsets (Task
1.1.2.3). 

It is important to note that with its focus on mathematical
descriptions for a catalogue of process models, this requirements
document is significantly different from a traditional Software
Requirements Specification (SRS) document.  Hence, this document does
not follow the IEEE Std 830-1998 template, and instead uses a process
category based layout that is summarized in Section~\ref{sec:layout}.
Moreover, this mathematical focus serves multiple audiences:
%
\begin{enumerate}
\item 
  This document provides guidance to the developers engaged in
  designing and implementing the HPC Core Framework and HPC Toolsets.
  To meet their needs, sufficient detail for each process model is
  provided in the form of a background discussion, supporting equations, 
  and references to relevant literature.  
\item 
  The document is also intended for ``domain scientists'' whose
  primary interest is in the processes themselves.  The presentation
  is intended to justify the choice of process models and their
  mathematical detail.
\item 
  Finally, the document is also intended for end users engaged in
  individual site applications.
\end{enumerate}
%
Over time this document will evolve into a comprehensive graded
presentation of models, from complex to simple, under a general
mathematical framework for each process category.  
%
%~\todo{Not sure what categories you are referring to here. -Finsterle}
Evolution of the list of processes is inevitable, and the modular
design of the HPC Simulator will easily accommodate the addition of
new process implementations. 



\subsection{Organization and Layout of this Document}
\label{sec:layout}

The remainder of this document is organized based on individual process category.
These include presentation of
\emph{Isothermal Flow Processes} in Section~\ref{sec:flow-processes}, 
\emph{Thermal Processes} in Section~\ref{sec:thermal-processes},
\emph{Transport Processes} in Section~\ref{sec:transport-processes}, and 
\emph{Biogeochemical Reaction Processes} in Section~\ref{sec:biogeochemical}. 
%\emph{Colloid Transport Processes} in Section~\ref{sec:colloids}, 
%\emph{Thermal Processes} in Section~\ref{sec:thermal}, 
%\emph{Geomechanical Processes} in Section~\ref{sec:geomechanics}, and 
%\emph{Source Terms}, in Section~\ref{sec:source-terms}.  
%
Each of these process categories may present multiple process models.  
%For example, biogeochemistry includes individual process models for sorption, 
%mineral precipitation-dissolution, microbially-mediated reactions, colloid
%generation, etc.  
%In addition, models of differing fidelity can be accommodated by the HPC code so, 
%for example, several models for sorption 
%(e.g., classical Kd, multicomponent-multisite ion exchange,
%non-electrostatic surface complexation, electrostatic surface
%complexation) are included.


In Section~\ref{sec:flow-processes}, \emph{Isothermal Flow Processes}, 
we consider models of fluid flow at constant temperature.
We limit ourselves to the case of a single fluid phase, i.e. no water-oil flows.
Based on the saturation levels we use one of two models:
fully saturated and partially saturated flows.
Further, use of Richards Equation requires certain assumptions on the gas phase not moving.

In Section~\ref{sec:thermal-processes}, \emph{Thermal Processes},
we discuss extension of the models of Section~\ref{sec:flow-processes}
to the case of changing temperature by adding a concervation of energy equation.


In Section~\ref{sec:transport-processes}, \emph{Transport Processes},
we consider models for transport, i.e. evolution of distributions, of solute species. 
Each of the species can be part of either gas, fluid or solid phase.
In the solid phase specied do not move.
In the gas phase species can only diffuse (due to the assumptions of Richards equation).
In the fluid phase the transport of the species is affected by the 
fluid flow as well as dispersion and diffusion processes. 
In this section we also consider a dual porocity models,
designed to capture the difference in fluid motion in the cracks and pores.  
 

In Section~\ref{sec:biogeochemical}, \emph{Biogeochemical Reaction Processes},
we consider a variety of biochemical reaction processes,
which from the point of view of model equations
convert one species into others and result in various heat sources.
  

The notational conventions and variables used throughout the document 
are summarized in Section~\ref{sec:notations}.




%--------------------------------------
\subsection{Variables and Notations}
\label{sec:notations}


% bold symbols
\def\bnabla{{\boldsymbol{\nabla}}}
\def\bg{{\boldsymbol{g}}}
\def\bq{{\boldsymbol{q}}}
\def\bx{{\boldsymbol{x}}}
\def\bJ{{\boldsymbol{J}}}
\def\K{{\mathbb K}}

% abbreviations
\def\Frac{\displaystyle \frac}

% units
\def\ucdot{{\,\cdot\,}}
\def\ukg{{\rm kg}}
\def\um{{\rm m}}
\def\us{{\rm s}}
\def\umol{{\rm mol}}
\def\upa{{\rm Pa}}


In this section we list all the variables used throughout the rest of the document.
In particular,
for multiple variables we use subscript 
$l$, $s$ and $g$
to indicate that the particular quantity is the property of the  
\emph{liquid}, \emph{solid} or \emph{gas}, respectively.
We also use subscripts $m$ and $f$
to indicate quantities that are attributed to the  
\emph{matrix}/pores and  \emph{fracture}, respectively.


\begin{center}
\begin{longtable}{cp{7cm}c}
\caption{List of global variables.} \label{table:flow-list-of-variables} \\

\multicolumn{1}{c}{Symbol} & \multicolumn{1}{c}{Meaning} & \multicolumn{1}{c}{Units} \\
\hline  \hline 
\endfirsthead

\multicolumn{3}{c}{{\tablename} \thetable{} -- Continued} \\
\multicolumn{1}{c}{Symbol} & \multicolumn{1}{c}{Meaning} & \multicolumn{1}{c}{Units} \\
\hline  \hline 
\endhead

\hline \multicolumn{3}{c}{{Continued on next page}} \\ 
\hline \hline 
\endfoot

\hline \hline
\endlastfoot

$C_i$      & concentration of $i$th species    &  $\umol\ucdot\um^{-3}$ \\
$e_l$      & enthalpy of the liquid        &  $\um$  \\
$\bg$      & gravity vector       &  $\um\ucdot\us^{-2}$  \\
$g$        & gravity magnitude    &  $\um\ucdot\us^{-2}$  \\
$h$        & hydrolic head        &  $\um$  \\
$\bJ$      & generic flux         &  $\umol\ucdot\um^{-2}\ucdot\us^{-1}$  \\
$\K$       & absolute permeability tensor & $\um^2$ \\
$k_{rl}$   & relative permeability&  $-$ \\
$p_l$      & liquid pressure      &  $\upa$ \\
$\bq$      & Darcy velocity       &  $\um\ucdot\us^{-1}$  \\
$S_s$      & specific storage     &  $\um^{-1}$  \\
$S_y$      & specific yield       &  $-$  \\
$s_l$      & liquid saturation    &  $-$ \\
$u_l$      & internal energy of the liquid   &  $\ukg\cdot\um^2\cdot \us^{-2}$ \\
$u_r$      & internal energy of the rock     &  $\ukg\cdot\um^2\cdot \us^{-2}$  \\
\hline
$\mu_l$    & liquid viscosity     &  $\upa\ucdot\us$ \\
$\rho_l$   & liquid density       &  $\ukg\ucdot\um^{-3}$ \\
$\phi$     & porosity             &  $-$  \\
$\phi_f$   & porosity of fracture &  $-$  \\
$\phi_m$   & porosity of matrix   &  $-$  \\
$\eta_l$   & molar liquid density &  $\umol\ucdot\um^{-3}$ \\
$\theta$   & volumetric water content  &  $\umol\ucdot\um^{-3}$ \\
$\theta_f$ & volumetric water content of fracture &  $\umol\ucdot\um^{-3}$ \\
$\theta_m$ & volumetric water content of matrix   &  $\umol\ucdot\um^{-3}$ \\

\end{longtable}
\end{center}








